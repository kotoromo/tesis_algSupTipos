\documentclass{article}
\usepackage{array}
\usepackage{graphicx}
\usepackage[spanish,es-noshorthands, es-lcroman]{babel}
\usepackage[utf8]{inputenc}
\usepackage{amsthm}
\usepackage{amsfonts}
\usepackage{amsmath}
\usepackage{amssymb}
\usepackage{enumerate}
\usepackage{amsmath}
\usepackage{calrsfs}
\usepackage{mathrsfs}
\usepackage{hyperref}
\usepackage{graphicx}
\usepackage{float}
\usepackage{tikz-cd}
\usepackage{todonotes}
\usepackage{tikz}
\usepackage{tikz-qtree}
\usepackage{pict2e}
\usepackage{subcaption}
\usepackage{wrapfig}
\usepackage{cite}
\usepackage[skins]{tcolorbox}
\usepackage{bussproofs}
\usepackage{bussproofs-extra}
\usepackage{bbold}
\usepackage{quiver}
\graphicspath{ {./img/} }
\newtcolorbox{sfwt}[2][]{%
  enhanced,colback=white,colframe=black,coltitle=black,
  sharp corners,boxrule=0.4pt,
  fonttitle=\itshape,
  attach boxed title to top left={yshift=-0.3\baselineskip-0.4pt,xshift=2mm},
  boxed title style={tile,size=minimal,left=0.5mm,right=0.5mm,
    colback=white,before upper=\strut},
  title=#2,#1
}

\newcommand{\overbar}[1]{\mkern 1.5mu\overline{\mkern-1.5mu#1\mkern-1.5mu}\mkern 1.5mu}

\DeclareMathOperator{\dom}{dom}
\DeclareMathOperator{\cod}{cod}
\DeclareMathOperator{\Id}{Id}
\DeclareMathOperator{\ran}{ran}
\DeclareMathOperator{\im}{ran}
\DeclareMathOperator{\cam}{cam}
\DeclareMathOperator{\sop}{Sop}
\DeclareMathOperator{\inr }{inr }
\DeclareMathOperator{\inl}{inl}
\DeclareMathOperator{\ind}{ind}

\graphicspath{ {./img/} }


\begin{document}

\theoremstyle{definition}
\newtheorem{definition}{Definición}[section]
\newtheorem{theorem}{Teorema}[section]
\newtheorem{proposition}{Proposición}[section]
\newtheorem{corollary}{Corolario}[theorem]
\newtheorem{lemma}[theorem]{Lema}
\newtheorem{remark}{Observación}
\newtheorem*{notation}{Notación}
\newtheorem{example}{Ejemplo}[section]
\newtheorem{exercise}{Juego}[section]
\newtheorem{axiom}{Axioma}

%%%%%%% DEFINITIONS %%%%%%%%%
\newcommand{\bb}[1]{\mathbb{#1}}
\newcommand{\set}[1]{\{#1\}}
\newcommand{\seq}[1]{\{#1\}_{n\in\bb{N}}}
\newcommand{\picopar}[1]{\langle #1 \rangle}
\newcommand{\card}[1]{\vert #1 \vert}
\newcommand{\RestrictTo}[1]{\restriction_{#1}}
\newcommand{\norm}[1]{\left\lVert{#1}\right\rVert}
\newcommand{\type}{\mathrm{type}}


\title{Álgebra Superior: Una perspectiva típica}
\author{Nicky García Fierros}

\maketitle
\tableofcontents

\section{Introducción}
\subsection{¿Teoría de tipos? ¿Y la teoría de conjuntos?}
\section{Parte I: Un poco de lógica}
\section{Introducción}
\subsection{Lógica como herramienta de razonamiento}

Seguramente el o la lectora ya habrá escuchado sobre cómo las matemáticas son "correctas" o inclusive se ha usado
un adjetivo más fuerte como "verdaderas" y quizás también el o la lectora habrá escuchado que todo problema en matemáticas tiene 
solución. Estas últimas afirmaciones son opiniones muy controversiales en tanto que existen resultados que ponen en duda su 
veracidad. 

En teoría de conjuntos existen resultados y problemas que rompen lo que llamaríamos sentido común: uno de ellos 
es el famoso teorema de Banach-Tarski que, dicho en términos coloquiales, te permite duplicar una esfera a partir de otra; \todo{Incluir un dibujo de la paradoja de Banach-Tarski}
otro problema, que es un directo contraejemplo a las afirmaciones hechas al principio del párrafo, es la hipótesis del continuo.

La hipótesis del continuo afirma que no existe un conjunto cuya cardinalidad está estrictamente entre los números enteros y 
los números reales (para fines del ejemplo basta pensar en cardinalidad como otra palabra para referirse a tamaño o cantidad 
de elementos). Está demostrado que esta última afirmación es independiente de los axiomas estándar de la teoría de conjuntos, y 
resulta que por motivos históricos muy razonables toda la matemática está fundamentada sobre esta axiomática estándar, 
por lo que en resumen, ¡existe un problema en matemáticas que no es resoluble!. 

Más que asustar al lector o a la lectora sobre su camino en matemáticas, el autor espera que estos ejemplos sirvan de motivación para 
seguir estudiando matemáticas y también como motivación para que el o la lectora se anime a explorar a las teorías donde surgen 
tan interesantes resultados.

Una vez dicho lo anterior, es claro también que debe de existir cierto grado de verdad en las matemáticas. Después de todo,
existen las computadoras, las cuales son máquinas que hacen operaciones matemáticas con el propósito de resolver problemas
muy reales (o ver videos de gatos en internet); existen los cohetes espaciales los cuales pueden viajar al espacio gracias
a los procesos matemáticos detrás de su diseño e implementación, y muchas otras cosas cuyo funcionamiento depende fuertemente
de las matemáticas.

La forma en que razonamos en matemáticas está muy cercana a la forma en que se razonan en otras disciplinas como la medicina,
el derecho, la ingeniería, la física, la economía, la sociología, la filosofía y demás; pues hay una presuposición de una 
aceptación de los principios básicos de la lógica, es decir, hay un cierto estándar de lo que es \textit{racional}. 
En las disciplinas mencionadas anteriormente se espera que aquellos quienes la practican puedan diferenciar entre un argumento 
racional con base en principios básicos o evidencia, y especulación y afirmaciones que de ninguna forma se siguen de la evidencia 
o los principios básicos.

Toda forma de investigación y razonamiento adecuado depende de la lógica tanto para descubrir cosas nuevas como para percatarse
de que se ha cometido un error y es necesario corregir algo. 

Por ejemplo, supóngase que se tiene un amigo en quien se confía mucho y suponemos que de contarle algún secreto éste no se 
lo revelaría a nadie. En virtud de lo anterior, decidimos confiarle un secreto muy vergonzoso: el día de ayer mojamos la cama 
soñando que estábamos jugando en la orilla del mar. Sin embargo, a lo largo del día descubrimos que ¡dicho amigo ha contado el 
secreto a varios de nuestros conocidos y conocidas!. Obviamente estamos que morimos de vergüenza y, aprendiendo nuestra lección 
notamos que hemos cometido un error en nuestra creencia: es falso que podemos confiar secretos en nuestro (ex)amigo. 

El autor cree
que con el ejemplo anterior se hace evidente que la lógica no solo tiene aplicaciones centrales en las ciencias, sociales o no,
sino que también en la vida diaria.

Surge de forma muy natural el cuestionarse ¿cómo es que exactamente una afirmación se sigue correctamente de otra?, o su 
contrapuesta: ¿exactamente cuándo ocurre que una proposición no se sigue de otra?; otra muy natural es ¿cuáles y por qué son 
las leyes de la lógica?. Estas preguntas son las que uno se hace cuando estudia lógica.

Curiosamente, al estudiar lógica uno hace uso de la misma lógica para estudiarla, pues se requiere tener una forma de razonar
que nos permita llegar a cosas que consideramos verdaderas. Regresando a matemáticas, la lógica ha tenido un papel tan
importante que se estudia a la lógica mediante técnicas propias de la matemática haciendo de la misma lógica una rama más de las
matemáticas coloquialmente denominada como \textit{lógica matemática}, la cual es una sub-rama de otra área de las matemáticas
denominada \textit{fundamentos de las matemáticas}, de la cual forman parte la teoría de conjuntos y la teoría homotópica de tipos. 

En las matemáticas estándar, aquella que se suele enseñar en las facultades, se hace uso de un "sabor" muy particular de la 
lógica, denominada lógica de primer orden y de la cual hablaremos a lo largo de esta sección, desarrollándola poco a poco, 
pero no necesariamente de forma minuciosa como se haría en un curso de lógica matemática. El motivo detrás de desarrollarla paso por 
paso es que el entendimiento de la lógica es primordial para estudiar matemáticas y para hacer matemáticas.

\section{Formalización de matemáticas en Agda}

\subsection{Introducción}
\todo{Meterle más paja a esto}
La teoría homotópica de tipos es un área de estudio de las matemáticas relativamente nueva. 
Esta área de estudio contempla herramientas de la teoría de los lenguajes de programación, el álgebra, la teoría de categorías, la lógica matemática y la topología.
El poder expresivo del lenguaje formal empleado por la teoría de tipos homotópica así como su fundamento teórico es tan expresivo y general que permite
ofrecer una teoría alternativa a la teoría de conjuntos para fundamentar las matemáticas. Dentro de las ventajas que brinda emplear este lenguaje
está la posibilidad de utilizar computadoras para verificar la correctud de demostraciones matemáticas.

Es importante notar que al ser ésta una teoría constructiva desde su concepción, técnicas propias que dependen de axiomas o teoremas no constructivos como lo son
la ley del tercer excluido, o el teorema de elección generalizado, no se encuentran disponibles en todos los contextos a diferencia de las "matemáticas clásicas".

En esta segunda parte del trabajo se explorarán de forma breve y concisa temas de la teoría homotópica de tipos con el objetivo
de proponer y dar una base teórica para una formalización del temario de álgebra superior.

\subsection{Teoría de tipos dependientes}
    \subsubsection{Juicios, contextos y derivaciones}
        En la teoría de tipos se emplea un lenguaje formal que está basado en la deducción natural pues es un sistema en el que se cuenta
        con reglas de inferencia que se pueden combinar para formar derivaciones. Las derivaciones nos importan porque son el principal
        mecanismo para producir \textit{términos} de un tipo determinado.
        
        Como es de esperarse del título que carga la teoría de tipos, un \textbf{tipo} es un objeto primitivo de la teoría de tipos de la misma fórma que un conjunto
        es un objeto primitivo de la teoría de conjuntos. Como podrá usted, lector o lectora, darse una idea desde el párrafo anterior, los tipos pueden tener (o no)
        términos. Como se mencionó antes, un término es el resultado de la aplicación de reglas de inferencia y, como el autor no desea arruinar el placentero proceso
        de entender a un nuevo objeto matemático, conforme avancemos en este trabajo floreceran distintas formas útiles de pensar a los tipos y sus términos.

        Entenderemos por una \textbf{derivación} a una sucesión de aplicaciones de \textbf{reglas de inferencia}.

        Comenzamos por definir precisamente qué es un juicio en este lenguaje.
        \todo{Argumentar sobre la necesidad del orden en los contextos y lo que significan los juicios para nosotros/ distincion entre proposiciones/enunciados}
        \begin{definition}[Juicios, contextos]
            Un \textbf{juicio} es alguna expresión de la forma:
            \begin{enumerate}
                \item $\Gamma \vdash A\ \type$ (Desde $\Gamma$ se deduce que $A$ es un tipo)
                \item $\Gamma \vdash a : A\ \type$ (Desde $\Gamma$ se deduce que $a$ es un término de tipo $A$)
                \item $\Gamma \vdash A \equiv B\ \type$ (Desde $\Gamma$ se deduce que $A$ es un tipo juiciosamente equivalente al tipo $B$)
                \item $\Gamma \vdash a \equiv b\ : A$ (Desde $\Gamma$ se deduce que los términos $a$ y $b$ de tipo $A$ son juiciosamente equivalentes)
            \end{enumerate}

            donde $\Gamma$ es una lista finita de declaraciones de variables tales que para cada $1 \leq k \leq n$
            se puede derivar el juicio
            $$
                x_1 : A_1, x_2 : A_2(x_1), \dots, x_k : A_k(x_1, x_2, \dots, x_{k-1}) \vdash A_{k+1}(x_1, x_2, \dots, x_{k-1}, x_k)\ \type
            $$
            y recibe el nombre de \textbf{contexto}; y lo que se encuentra a la derecha del símbolo $\vdash$ 
            (léase "desde \_ se deduce \_") recibe el nombre de \textbf{tésis de juicio}.
        \end{definition}

        Los contextos, de forma análoga a su rol en el cálculo de secuentes, denotan los supuestos que se están considerando para obtener la tésis de juicio.
        En tanto que los elementos potencialmente pueden ser suposiciones que carecen de fundamento previamente derivado se les suelen llamar \textit{variables}.
        Los juicios los pensamos como hechos, a diferencia de las proposiciones; las cuales potencialmente son verdaderas o falsas.
        Alternativamente llamaremos \textbf{elementos} a los términos de un tipo dado, de modo que un juicio $a : A$ se puede leer como $a$ es un elemento de tipo $A$.
        \begin{remark}
            Observe que nuestra definición de contexto permite la existencia de un contexto vacío pues por un argumento de vacuidad se verifica la
            satisfacibilidad de la propiedad de un contexto.
        \end{remark}
        \begin{remark}
            Obsérvese que la condición impuesta sobre un contexto se puede verificar de forma recursiva o inductiva:
            \begin{itemize}
                \item El caso base es mostrar que $x_1 : A_1$ se deduce desde el contexto vacío. Para afirmar que $x_1 : A_1$ es un juicio válido se debe haber deducido (o supuesto)
                que $A_1$ es un tipo en el contexto vacío.
                \item La clausula inductiva es codificada por la propiedad que define a un contexto.
            \end{itemize}

            Para verificar de forma recursiva que una lista de declaraciones de la forma 
            $$
                x_1 : A_1, x_2 : A_2(x_1), \dots, x_k : A_k(x_1, x_2, \dots, x_{k-1}) \vdash x_{k+1} : A_{k+1}(x_1, \dots, x_{k-1}, x_k)\ \type
            $$
            es un contexto basta probar que una lista de declaraciones de la forma
            $$
            x_1 : A_1, x_2 : A_2(x_1), \dots, x_k : A_{k-2}(x_1, x_2, \dots, x_{k-2}) \vdash x_{k} : A_{k}(x_1, \dots, x_{k-1})\ \type
            $$
            y así de forma sucesiva hasta dar con el caso base.
        \end{remark}

        \begin{definition}[Derivación]
            Una \textbf{derivación} es un árbol finito con raíz en el que cada vértice es una regla de inferencia válida.
            A la raíz del árbol se le llama \textbf{conclusión} y a las hojas \textbf{hipótesis}.
        \end{definition}

        Nos reservamos el derecho de poder definir nuevas reglas de inferencia a partir de otras, y diremos que estas nuevas reglas son \textbf{derivables}.
    
    \subsubsection{Familias de tipos}
        Una idea universal bastante útil es la de un "agrupamiento de agrupamientos"; ejemplos clásicos de este patrón de pensamiento son
        las familias de conjuntos; en teoría de conjuntos; y los enunciados; en lógica de primer órden. En la teoría de tipos dependientes
        de Per Martin-Löf contamos con un marco de trabajo que engloba esta idea, la cual es la de una \textit{familia de tipos}.

        \begin{definition}[Familia de tipos]\label{def:familia_tipos}
            Si $A$ es un tipo en un contexto $\Gamma$, una \textbf{familia de tipos} $B(x)$ es un tipo en el contexto $\Gamma, x : A$ (o también diremos
            que $B(x)$ es un \textbf{tipo indizado sobre} $A$ en el contexto $\Gamma$) y
            escribimos formalmente este hecho como
            $$
                \Gamma, x : A \vdash B(x)\ \type
            $$

            y en su forma de regla de inferencia podemos \textbf{introducirla} como
            \begin{prooftree}
                \AxiomC{$\Gamma \vdash x : A$}
                \AxiomC{$\varnothing \vdash A\ \type$}
                \BinaryInfC{$B(x)\ \type$}
            \end{prooftree}

            Por comodidad se suele omitir el contexto vacío y solamente se escribe la tésis de juicio, de modo que escribimos:

            \begin{prooftree}
                \AxiomC{$\Gamma \vdash x : A$}
                \AxiomC{$A\ \type$}
                \BinaryInfC{$B(x)\ \type$}
            \end{prooftree}

            o si damos por obvio que $A$ tiene que ser un tipo para que el juicio $\Gamma \vdash x : A$ sea válido podemos solamente convenir escribir
            \begin{prooftree}
                \AxiomC{$\Gamma \vdash x : A$}
                \UnaryInfC{$B(x)\ \type$}
            \end{prooftree}
        
            Por conveniencia y claridad, a partir de este punto emplearemos las convenciones de escritura que nos permiten obviar cosas a menos de que sea
            necesario para esclarecer.
        \end{definition}

        \begin{remark}
            Resulta bastante útil pensar a una familia de tipos como un tipo que varía según los términos de otro tipo. 
            Es decir, si abusamos de notación, podemos pensar a una familia de tipos como una función 
            \begin{align*}
                \mathrm{Term(A)} &\rightarrow \mathrm{Types}\\
                x : A &\mapsto B(x)\ \type
            \end{align*}
            Un un futuro no muy lejano se exhibirá cómo expresar este hecho de manera formal dentro del lenguaje de la teoría de tipos dependiente.
        \end{remark}

        Como es de esperarse que de una colecciones de colecciones podamos tomar \textit{una parte}, análogamente de una familia de tipos
        podemos considerar lo que llamaremos una \textbf{sección}.

        \begin{definition}[Sección de una familia de tipos]
            Si $B$ es una familia de tipos sobre $A$ en el contexto $\Gamma$, diremos que una \textbf{sección} de $B$ es un término $b(x) : B(x)$ en un contexto
            $\Gamma, x : A$. En símbolos:
            $$
                \Gamma, x : A \vdash b(x) : B(x)
            $$
            La \textbf{regla de introducción} asociada entonces es:

            \begin{prooftree}
                \AxiomC{$\Gamma \vdash x : A$}
                \AxiomC{$\Gamma, x : A \vdash B(x)\ \type$}
                \BinaryInfC{$\Gamma \vdash b(x) : B(x)$}
            \end{prooftree}
            Y podemos entenderla como:
            \textit{Si podemos deducir del contexto $\Gamma$ que $x$ es un término de tipo $A$ y que $B$ es una familia de tipos sobre $A$, 
            entonces podemos deducir desde $\Gamma$ que $b(x) : B(x)$ es una sección de $B$}.
        \end{definition}

        \begin{remark}
            Nótese que tanto el término como el tipo dependen del término 
            $x : A$, de modo que abusando de la notación podemos pensar a este 
            proceso como una función
            \begin{align*}
                \mathrm{Term(A)} \times (\mathrm{Term(A)} \rightarrow \mathrm{Types}) &\rightarrow \mathrm{Term}(B(x))\\
                \picopar{x : A\ ,\ x : A \mapsto B(x)\ \type}\mapsto b(x) : B(x)
            \end{align*}
        \end{remark}
        
    \subsubsection{Clases de reglas de inferencia}
        Las siguientes reglas de inferencia describen de forma explicita las 
        suposiciones de que hicimos en la definición \ref{def:familia_tipos}.
        Es de esperarse que, si tenemos en un contexto las variables 
        $A\ \type, x : A$, entonces desde ese mismo contexto podamos deducir
        $A\ \type$ y $x : A$ por separado.
        \begin{center}
            \AxiomC{$\Gamma, x : A \vdash B(x)\ \type$}
            \UnaryInfC{$\Gamma \vdash A\ \type$}
            \DisplayProof
            \hskip 1.5em
            \AxiomC{$\Gamma \vdash A \equiv B\ \type$}
            \UnaryInfC{$\Gamma \vdash A\ \type$}
            \DisplayProof
            \hskip 1.5em
            \AxiomC{$\Gamma \vdash A \equiv B\ \type$}
            \UnaryInfC{$\Gamma \vdash B\ \type$}
            \DisplayProof
        \end{center}
        
        \begin{center}
            \AxiomC{$\Gamma \vdash a \equiv b : A$}
            \UnaryInfC{$\Gamma \vdash a : A$}
            \DisplayProof
            \hskip 1.5em
            \AxiomC{$\Gamma \vdash a \equiv b : A$}
            \UnaryInfC{$\Gamma \vdash b : A$}
            \DisplayProof
            \hskip 1.5em
            \AxiomC{$\Gamma \vdash a : A$}
            \UnaryInfC{$\Gamma \vdash A\ \type$}
            \DisplayProof
        \end{center}

        En tanto que es de interés que la noción de \textit{ser juiciosamente iguales} sea una buena noción de equivalencia, es
        de esperarse que se postulen reglas que testifican que esta noción satisface los axiomas de una relación de equivalencia.

        \begin{center}
            \AxiomC{$\Gamma \vdash A\ \type$}
            \UnaryInfC{$\Gamma \vdash A\equiv A\ \type$}
            \DisplayProof
            \hskip 1.5em
            \AxiomC{$\Gamma \vdash A \equiv B\ \type$}
            \UnaryInfC{$\Gamma \vdash B \equiv A\ \type$}
            \DisplayProof
            %\hskip 1.5em
            %\AxiomC{$\Gamma \vdash A \equiv B\ \type$}
            %\AxiomC{$\Gamma \vdash B \equiv C\ \type$}
            %\BinaryInfC{$\Gamma \vdash A \equiv C\ \type$}
            %\DisplayProof
        \end{center}

        \begin{prooftree}
            \AxiomC{$\Gamma \vdash A \equiv B\ \type$}
            \AxiomC{$\Gamma \vdash B \equiv C\ \type$}
            \BinaryInfC{$\Gamma \vdash A \equiv C\ \type$}
        \end{prooftree}

        \begin{center}
            \AxiomC{$\Gamma \vdash a : A$}
            \UnaryInfC{$\Gamma \vdash a \equiv a : A$}
            \DisplayProof
            \hskip 1.5em
            \AxiomC{$\Gamma \vdash a \equiv b: A$}
            \UnaryInfC{$\Gamma \vdash b \equiv a : A$}
            \DisplayProof
            \hskip 1.5em
            \AxiomC{$\Gamma \vdash a \equiv b : A$}
            \AxiomC{$\Gamma \vdash b \equiv c : A$}
            \BinaryInfC{$\Gamma \vdash a \equiv c : A$}
            \DisplayProof
        \end{center}

        También es de esperarse que, si se tienen que dos tipos son juiciosamente equivalentes, y puedes deducir una tésis de juicio $\mathfrak{T}$
        a partir de una variable, entonces al intercambiar el tipo sobre el que tomas la variable por su equivalente la misma tésis de juicio
        debería poder deducirse.

        \begin{center}
            \AxiomC{$\Gamma \vdash A \equiv B \ \type$}
            \AxiomC{$\Gamma, x : A, \Theta \vdash \mathfrak{T}$}
            \BinaryInfC{$\Gamma, x : B, \Theta \vdash \mathfrak{T}$}
            \DisplayProof
        \end{center}
        Donde $\Theta$ es una extensión cualquiera del contexto $\Gamma, x : A$.
        Por ejemplo en el caso en que $\mathfrak{T}$ es $C(x)\ \type$ tenemos
        \begin{center}
            \AxiomC{$\Gamma \vdash A \equiv B \ \type$}
            \AxiomC{$\Gamma, x : A, \Theta \vdash C(x)\ \type$}
            \BinaryInfC{$\Gamma, x : B, \Theta \vdash C(x)\ \type$}
            \DisplayProof
        \end{center}
        
        En general, el concepto de sustituir es uno muy importante en las 
        estructuras de pensamiento humanas, por lo que es de esperarse que dicho 
        concepto también esté persente en esta teoría.

        Al tener ya una noción de igualdad, podemos comenzar a hacernos 
        preguntas sobre sustituciones de elementos en otros elementos.

        Consideremos una sección $f(x)$ de una familia de tipos $B(x)$ indizado 
        por $x : A$ en un contexto $\Gamma$. Al ser que $f(x)$ como expresión
        contiene al menos una referencia a $x$, y $f(x) : B(x)$ entonces
        al sustituir cada referencia de $x$ por algún $a : A$ de forma 
        simultánea sobre $f(x) : B(x)$ debemos esperar que $f[a/x]$ sea un
        elemento de $B[a / x]$. 

        En general, \textbf{la regla de sustitución}

        \begin{center}
            \AxiomC{$\Gamma \vdash a : A$}
            \AxiomC{$\Gamma, x : A, \Theta \vdash \mathfrak{T}$}
            \BinaryInfC{$\Gamma, \Theta[a/x] \vdash \mathfrak{T}[a/x]$}
            \DisplayProof
        \end{center}

        también nos permite sustituir de forma simultánea sobre un contexto dado.
        Es importante mencionar que el orden de los elementos en un contexto es
        escencial, pues $\Gamma, x : A, \Theta$ no es lo mismo que 
        $\Gamma, \Theta, x : A$. El orden es indicativo de cierta potencial
        dependencia entre elementos del contexto.

        Por ejemplo, si

        $$
        \Gamma, x : A, s : S, m : M \vdash C(x)\ type
        $$

        entonces por la regla de sustitución, dado $\Gamma \vdash a : A$ 
        tendríamos

        $$
        \Gamma, s : S[a/x], m : M[a/x] \vdash C[a/x]\ type
        $$

        donde potencialmente los tipos de $s, m$ sean distintos. 
        Por otro lado si

        $$
        \Gamma, s : S, m : M, x : A \vdash C(x)\ type
        $$

        entonces la regla de sustitución sólo nos permite asegurar que

        $$
        \Gamma, s : S, m : M \vdash C[a/x]\ type
        $$

        y los tipos de $s$ y $m$ no sufren cambios.

        Es de esperarse que, si se tiene que dos elementos son juiciosamente
        iguales, entonces la sustitución respeta esta igualdad juiciosa.

        \begin{center}
            \AxiomC{$\Gamma \vdash a \equiv a' : A$}
            \AxiomC{$\Gamma, x : A, \Delta \vdash B\ \type$}
            \BinaryInfC{$\Gamma, \Delta[a/x] \vdash B[a/x] \equiv B[a'/x]$}
            \DisplayProof
        \end{center}

        \begin{center}
            \AxiomC{$\Gamma \vdash a \equiv a' : A$}
            \AxiomC{$\Gamma, x : A, \Delta \vdash b : B\ \type$}
            \BinaryInfC{$\Gamma, \Delta[a/x] \vdash b[x/a] : B[a/x] \equiv b[x/a] : B[a'/x]$}
            \DisplayProof
        \end{center}

        A partir de este momento acordamos en denotar a $b[x/a]$ por simplemente
        $b(a)$, y a $B[x/a]$ por $B(a)$ a menos que sea necesario emplear la
        notación usual de sustitución.
        
        Consideremos la siguente regla

        \begin{center}
            \AxiomC{$\Gamma \vdash A\ \type$}
            \AxiomC{$\Gamma, \Theta \vdash \mathfrak{T}$}
            \BinaryInfC{$\Gamma, x  : A, \Theta \vdash \mathfrak{T}$}
            \DisplayProof
        \end{center}

        A primer vistazo parece ser que la regla nos quiere decir que si podemos 
        derivar una tésis de juicio $\mathfrak{T}$ desde un contexto $\Gamma$, 
        entonces al introducir una variable no presente en $\mathfrak{T}$ ni en 
        $\Gamma$ (una variable libre) podamos deducir exactamente lo mismo. 

        Sin embargo esta no es toda la historia. Una pregunta natural que surge
        es \textit{¿qué ocurre con la nueva dependencia agregada sobre la 
        variable $x: A$?}. Por ejemplo, consideremos que desde un contexto
        $\Gamma$ podemos deducir que $A\ \type$ y $B\ \type$. ¡Entonces la regla
        anterior nos dice que podemos deducir que $B$ es una familia sobre $A$!

        \begin{center}
            \AxiomC{$\Gamma \vdash A\ \type$}
            \AxiomC{$\Gamma \vdash B\ \type$}
            \BinaryInfC{$\Gamma, x : A \vdash B\ \type$}
            \DisplayProof
        \end{center}

        En tanto que estamos agregando hipótesis adicionales a una derivación,
        decimos que estamos \textit{debilitando} la conclusión. De ahí que el
        nombre de la regla sea \textbf{\textit{weakening}} o 
        \textbf{\textit{regla de debilitamiento}}.

        La \textbf{regla de introducción de variables}, o también conocida como
        \textbf{regla del elemento genérico}, es un caso particular de la regla 
        de debilitamiento, en tanto que si la tésis de juicio es $A\ \type$ y 
        $\Theta$ es vacío, entonces podemos derivar

        \begin{center}
            \AxiomC{$\Gamma \vdash A\ \type$}
            \UnaryInfC{$\Gamma, x : A \vdash x : A$}
            \DisplayProof
        \end{center}

        Como ejemplos de derivaciones se presentan a continuación algunas
        reglas derivables útiles desde las reglas discutidas anteriormente:

        \begin{theorem}[sustitución de variables por otras]
            Sean $\Gamma$ y $\Theta$ contextos y $\mathfrak{T}$ una tésis de
            juicio tales que 
            $$
                \Gamma, x : A, \Theta \vdash \mathfrak{T}
            $$

            Entonces se puede deducir que

            $$
                \Gamma, x' : A, \Theta[x'/x] \vdash \mathfrak{T}[x/x']
            $$
        \end{theorem}
        \begin{proof}
            Tengo que escribirla gg
        \end{proof}

        \begin{theorem}[regla de intercambio del orden de variables]
            Sean $\Gamma$ y $\Theta$ contextos tales que
            $$
                \Gamma, x : A, y : B, \Theta \vdash \mathfrak{T}
            $$

            Entonces se tiene que desde el contexto
            $$
                \Gamma, y : B, x : A, \Theta
            $$
            se deduce el mismo juicio $\mathfrak{T}$.
        \end{theorem}
        \begin{proof}
            Tengo que escribirla gg
        \end{proof}

        \begin{theorem}[relga de conversión de elementos]
            Sea $\Gamma$ un contexto tal que de este se deduce
            $$
                \Gamma \vdash A \equiv A'\ \type    
            $$
            $$
                \Gamma \vdash a : A
            $$
            Entonces se puede deducir $\Gamma \vdash a : A'$.
        \end{theorem}
        \begin{proof}
            Tengo que escribirla gg
            \begin{center}
                \AxiomC{$\Gamma \vdash A \equiv A'\ \type$}    
                \AxiomC{$\Gamma \vdash a : A$}
                \BinaryInfC{$\Gamma \vdash a : A'$}
                \DisplayProof
            \end{center}
        \end{proof}

        \begin{theorem}[regla de congruencia para la conversión de elementos]
            Sea $\Gamma$ un contexto del cual se deduce 
            $\Gamma \vdash A \equiv A'$ y $\Gamma \vdash a \equiv b : A$.
            Entonces se puede deducir que $\Gamma \vdash a \equiv b : A'$.
        \end{theorem}

    \subsection{Tipos primitivos}
        Ya que contamos con un minimo fundamento sobre el cual poder construir
        tipos, procedemos a discutir sobre aquellos tipos que el sistema
        permite construir desde un contexto vacío. Estos tipos formarán los
        bloques básicos sobre los que haremos las construcciones de nuevos
        tipos y más aún, serán de gran utilidad para comenzar a darnos una idea
        de cómo codificar objetos matemáticos en este lenguaje.
    \subsubsection{Funciones dependientes}
        Una función dependiente podemos pensarla como aquella tal que permite
        que el codominio varíe en función de un elemento del dominio. En
        la teoría de conjuntos se presenta una construcción semejante, y es la
        del producto generalizado. Recordando, el producto generalizado de una
        familia indizada es
        $$
            \prod_{i \in \Gamma} X_i := 
            \{f : \Gamma \rightarrow \bigcup\limits_{i \in \Gamma} X_i\ 
                \vert\ \forall i \in \Gamma\ f(i) \in X_i\}
        $$

        de modo que un elemento $f$ del producto cartesiano es una función que
        dibuja una serie de posibilidades para el valor que puede tomar 
        $f(i) \in X_i$. Esta misma situación se nos presentó al introducir las
        familias de tipos,
        $$
            \mathrm{Ctx}, i : \Gamma \vdash X(i)\ \type
        $$
        $$
            \mathrm{Ctx}, i : \Gamma \vdash f(i) : X(i)
        $$
        de modo que $f : X$ es una función dependiente.
        \begin{definition}[tipo de funciones dependientes]
            La \textbf{regla de formación} del tipo de funciones dependientes 
            establece que la existencia de una familia de tipos es suficiente
            para obtener un tipo de funciones dependientes:
            \begin{center}
                \AxiomC{$\Gamma, x : A \vdash B(x)\ \type$}
                \UnaryInfC{$\Gamma \vdash \prod_{(x : A)}B(x)\ \type$}
                \DisplayProof
            \end{center}
            Además, la formación del tipo producto es congruente con la igualdad
            juiciosa, esto es,
            \begin{center}
                \AxiomC{$\Gamma \vdash A \equiv A'\ \type$}
                \AxiomC{$\Gamma x : A \vdash B(x) \equiv B'(x)\ \type$}
                \BinaryInfC{$\Gamma \vdash \prod_{(x: A)} B(x)\ \type 
                    \equiv \prod_{(x:A')}B(x)\ \type$}
                \DisplayProof
            \end{center}

            La \textbf{regla de introducción} establece que los elementos del
            tipo de funciones dependientes son exactamente las funciones que
            asignan a un término "índice" de la familia a una función

            \begin{center}
                \AxiomC{$\Gamma, x : A \vdash b(x):B(x)$}
                \UnaryInfC{$\Gamma \vdash \lambda x\ .\ b(x) : B(x)$}
                \DisplayProof
            \end{center}
            Más aún, postulamos la congruencia de esta regla ante la igualdad
            juiciosa:
            \begin{center}
                \AxiomC{$\Gamma, x : A \vdash b(x) \equiv b'(x) : B(x)$}
                \UnaryInfC{$\Gamma \vdash \lambda x\ .\ b(x) \equiv \lambda x\ .\ b'(x) : \prod_{(x : A)} B(x)$}
                \DisplayProof
            \end{center}

            La \textbf{regla de eliminación} del tipo de funciones dependientes,
            como es de esperarse, nos permite eliminar de un árbol de deducción
            un término del tipo de funciones dependientes siempre y cuando
            podamos evaluarlo para obtener un término del tipo resultante:
            \begin{center}
                \AxiomC{$\Gamma \vdash f : \prod_{(x : A)} B(x)$}
                \UnaryInfC{$\Gamma, x : A \vdash f(x) : B(x)$}
                \DisplayProof
            \end{center}

            La \textbf{regla de cómputo} del tipo de funciones dependientes
            postula que la evaluación de un término de $\prod_{(x : A)} B(x)$ es
            simplemente evaluar el término dado en $A$ en $b(x) : B(x)$,
            semejante a la reducción $\beta$ del cálculo lambda:
            \begin{center}
                \AxiomC{$\Gamma, a : A\vdash b(a) : B(a)$}
                \UnaryInfC{$\Gamma, x : A \vdash (\lambda x\ .\ b(x))(a) \equiv b(a) : B(x)$}
                \DisplayProof
            \end{center}
            Es importante remarcar que, es necesario garantizar la existencia
            del elemento en el contradominio que le corresponde a $a : A$ para
            poder aplicar la regla de cómputo.

            Por otro lado, \textbf{la regla $\eta$} nos asegura que los elementos de un
            tipo de funciones dependientes son exactamente funciones.
            \begin{center}
                \AxiomC{$\Gamma \vdash b : \prod_{(x : A)} B(x)$}
                \UnaryInfC{$\Gamma \vdash \lambda x\ .\ b(x) \equiv b : \prod_{(x : A)} B(x)$}
                \DisplayProof
            \end{center}
        \end{definition}

        \begin{remark}
            Análogamente a su simíl en conjuntos, una familia de tipos involucra
            una elección, en este caso de un $b(x) : B(x)$ dado un $x : A$.
        \end{remark}
        \begin{remark}
            Observe que la regla de cómputo y la regla $\eta$ son inversas 
            mutuas.
        \end{remark}

        Observe que, si tratamos con una familia de tipos constante; esto es que
        el tipo codominio no varia según el término índice; tenemos una función.
        Las reglas que definen al tipo de funciones dependientes se simplifican
        entonces:
        \begin{center}
            \AxiomC{$\Gamma \vdash A\ \type$}
            \AxiomC{$\Gamma \vdash B\ \type$}
            \RightLabel{(weakening)}
            \BinaryInfC{$\Gamma, x : A \vdash B\ \type$}
            \UnaryInfC{$\Gamma \vdash \prod_{(x : A)} B\ \type$}
            \DisplayProof
        \end{center}
        De modo que, ante una situación como la anterior la regla de 
        introducción nos diría que las funciones son exactamente las
        abstracciones lambdas sobre un tipo en función de un término:
        \begin{center}
            \AxiomC{$\Gamma x : A \vdash b(x) : B$}
            \UnaryInfC{$\Gamma \vdash \lambda x\ .\ b(x) : \prod_{(x : A)} B$}
            \DisplayProof
        \end{center}

        La regla de eliminación nos dice exactamente lo que esperaríamos de un
        tipo que codifica una función:
        \begin{center}
            \AxiomC{$\Gamma \vdash f : \prod_{(x : A)} B$}
            \UnaryInfC{$\Gamma, x : A \vdash f(x) : B$}
            \DisplayProof
        \end{center}
        Si evaluamos una función $f$ con dominio en $A$ y codominio en $B$ en un
        elemento $x : A$ del dominio, entonces $f(x) : B$.

        Así, mediante una regla de derivación consolidamos nuestra definición
        del tipo de funciones o equivalentemente llamado tipo flecha:
        
        \begin{definition}[tipo de funciones]
            El tipo de funciones de un tipo $A$ en un tipo $B$ se define como
            a continuación:
            \begin{center}
                \AxiomC{$\Gamma \vdash A\ \type$}
                \AxiomC{$\Gamma \vdash B\ \type$}
                \BinaryInfC{$\Gamma, x : A \vdash B\ \type$}
                \UnaryInfC{$\Gamma \vdash \prod_{(x : A)} B\ \type$}
                \UnaryInfC{$\Gamma \vdash A \rightarrow B := \prod_{(x : A)} B\ \type$}
                \DisplayProof
            \end{center}
        \end{definition}
        En general, dada una construcción podemos crear una definición con base
        en el resultado final. Para ello, conveniremos en el símbolo $:=$ para
        denotar que se está realizando una definición. Como es de esperarse,
        las mismas reglas que aplicaban para el tipo de funciones dependientes
        aplican para nuestra definición del tipo de funciones:

        \begin{center}
            \AxiomC{$\Gamma \vdash A\ \type$}
            \AxiomC{$\Gamma \vdash B\ \type$}
            \BinaryInfC{$\Gamma \vdash A \rightarrow B\ \type$}
            \DisplayProof
            \hskip 1.5 em
            \AxiomC{$\Gamma, x : A \vdash f(x) : B$}
            \UnaryInfC{$\Gamma \vdash \lambda\ x\ .\ f(x) : A \rightarrow B$}
            \DisplayProof
        \end{center}
        \begin{center}
            \AxiomC{$\Gamma \vdash A \equiv A' \type$}
            \AxiomC{$\Gamma \vdash B \equiv B' \type$}
            \BinaryInfC{$\Gamma \vdash A \rightarrow B \equiv A' \rightarrow B'\ \type$}
            \DisplayProof
        \end{center}
        \begin{center}
            \AxiomC{$\Gamma \vdash f: A \rightarrow B$}
            \UnaryInfC{$\Gamma, x : A \vdash f(x) : B$}
            \DisplayProof
            \hskip 1.5 em
            \AxiomC{$\Gamma \vdash f \equiv g : A \rightarrow B$}
            \UnaryInfC{$\Gamma, x : A \vdash f(x) \equiv g(x) : B$}
            \DisplayProof
        \end{center}
        \begin{center}
            \AxiomC{$\Gamma, x : A \vdash f(x) : B$}
            \UnaryInfC{$\Gamma \vdash \lambda x\ .\ f(x) : A \rightarrow B$}
            \DisplayProof
            \hskip 1.5 em
            \AxiomC{$\Gamma \vdash f : A \rightarrow B$}
            \UnaryInfC{$\Gamma \vdash \lambda x\ .\ f(x) \equiv f : A \rightarrow B$}
            \DisplayProof
        \end{center}
        \begin{center}
            \AxiomC{$\Gamma \vdash B\ \type$}
            \AxiomC{$\Gamma, a : A \vdash f(a) : B$}
            \BinaryInfC{$\Gamma, a : A \vdash (\lambda\ x\ .\ f(x))(a) \equiv f(a) : B$}
            \DisplayProof
        \end{center}
    Para terminar con esta subsección se presentan a continuación algunas
    construcciones útiles con tipos flecha.
    \subsubsection*{Algunas construcciones útiles con el tipo flecha}
    \textbf{La flecha identidad.}\newline
    Deseamos definir un objeto que codifique a la flecha identidad. Sabemos que
    la flecha identidad es tal que para todo objeto perteneciente al dominio se
    corresponde a si mismo bajo esta flecha. En general, algo a observar de lo
    anterior es que en principio el dominio y contradominio de la flecha
    identidad puede ser el que sea mientras exista. De esta forma, comenzamos
    nuestra construcción postulando que de tener un tipo en algún contexto
    podemos entonces construir este objeto.
    $$
        \Gamma \vdash A\ \type
    $$
    Luego, aplicando la regla de introducción de variables podemos obtener de
    lo anterior lo siguiente:
    \begin{center}
        \AxiomC{$\Gamma \vdash A\ \type$}
        \UnaryInfC{$\Gamma, x : A \vdash x : A$}
        \DisplayProof
    \end{center}
    Por la conclusión anterior, podemos entonces aplicar la regla de 
    introducción del tipo flecha:
    \begin{center}
        \AxiomC{$\Gamma \vdash A\ \type$}
        \UnaryInfC{$\Gamma, x : A \vdash x : A$}
        \UnaryInfC{$\Gamma \vdash \lambda\ x\ .\ x : A \rightarrow A$}
        \DisplayProof
    \end{center}
    Para así poder concluir nuestra definición:
    \begin{center}
        \AxiomC{$\Gamma \vdash A\ \type$}
        \UnaryInfC{$\Gamma, x : A \vdash x : A$}
        \UnaryInfC{$\Gamma \vdash \lambda\ x\ .\ x : A \rightarrow A$}
        \UnaryInfC{$\Gamma \vdash \Id_A := \lambda\ x\ .\ x : A \rightarrow A$}
        \DisplayProof
    \end{center}

    Así como partimos desde arriba la construcción, es usual comenzar desde
    abajo e ir construyendo el árbol desde abajo hacia arriba, rellenando huecos
    que vayan surgiendo para llegar a lo deseado. Utilizaremos esta técnica para
    definir la operación de composición a modo de exhibición.\newline\newline
    \textbf{La composición de flechas.}\newline\todo{Revisar como introducir esto. Creo que me estoy comiendo pasos.}
    Bajo el paradigma de \textit{tipos como proposiciones y términos como 
    demostraciones}, un buen punto de partida es plasmar en un tipo lo que se 
    desea demostrar. En este caso, queremos demostrar que dadas dos flechas 
    $f : A \rightarrow B$ y $g : B \rightarrow C$ podemos definir una flecha 
    $\mathrm{comp}(f, g) : A \rightarrow C$ que sea tal que 
    $\mathrm{comp}(f, g)(x) := g(f(x))$. 
    Así, el tipo que deseamos demostrar que tiene al menos un término es:
    \begin{center}
        \AxiomC{$\Gamma \vdash ? : (B^A \rightarrow C^B) \rightarrow C^A$}
        \DisplayProof
    \end{center}

    Sin embargo desconocemos la forma explícita que tiene dicho objeto.
    Desde luego tiene que ser un término de tipo flecha, y la regla de
    introducción de estos objetos nos dice que en general tienen la forma
    $\lambda\ x\ .\ h(x)$, por lo que nuestro objeto deseado no será la
    excepción. 
    \begin{center}
        \AxiomC{$\Gamma \vdash ?$}
        \UnaryInfC{$\Gamma \vdash \lambda\ x\ .\ ? : (B^A \rightarrow C^B) \rightarrow C^A$}
        \DisplayProof
    \end{center}
    De nuevo, como nuestro resultado es un término de tipo flecha, el paso
    inmediato anterior debe ser una aplicación de la regla de introducción de
    este tipo. Esta regla de introducción solicita de nosotros una 
    \textit{entrada}: un término de tipo $C^A$ que debe ser deducido a partir 
    del tipo $B^A \rightarrow C^B$.
    \begin{center}
        \AxiomC{$\Gamma, ? : B^A \rightarrow C^B \vdash ? : C^A$}
        \UnaryInfC{$\Gamma \vdash \lambda\ x\ .\ ? : (B^A \rightarrow C^B) \rightarrow C^A$}
        \DisplayProof
    \end{center}
    
    Ahora bien, debemos probar que en efecto $\Gamma, x : A \vdash g(f(x)) : C$.
    Por la regla de elemento genérico fácilmente podemos deducir desde 
    $A\ \type$ que $\Gamma, x : A \vdash x : A$. Por otro lado, si $f$ y $g$ son
    elementos de tipo flecha, entonces podemos aplicar la regla de eliminación
    para obtener desde $g$ un elemento de tipo $C$ suponiendo que en efecto
    $f(x) : B$.
    \begin{center}
        \AxiomC{$\Gamma \vdash A\ \type$}
        \UnaryInfC{$\Gamma, x \vdash x : A$}
        \AxiomC{$\Gamma, ? \vdash f(x) : B$}
        \UnaryInfC{$\Gamma, f(x) : B \vdash g(f(x)) : C$}
        \BinaryInfC{$\Gamma, x : A \vdash g(f(x)) : C$}
        \UnaryInfC{$\Gamma \vdash \lambda\ x\ .\ g(f(x)) : A \rightarrow C$}
        \DisplayProof
    \end{center}

    \subsubsection{Productos dependientes}
    \subsubsection{Coproducto}
    % tengo que evaluar si esto si va.... se puede quedar como comentario.
    \subsubsection{Tipos inductivos}
    
    \subsubsection{Tipos de identidad}

    \subsubsection{Universos de tipos}
        La teoría de conjuntos permite tener conjuntos cuyos elementos son 
        conjuntos. De manera similar, la teoría de tipos ofrece un mecanismo
        para definir tipos cuyos términos son también tipos. Los 
        \textbf{universos} de tipos consisten de un tipo $\mathfrak{U}$ junto
        con una familia de tipos $\mathfrak{T}$ definida sobre $\mathfrak{U}$.
        La idea es pensar que dado $X : \mathfrak{U}$, $\mathfrak{T}(X)$ es
        una "\textit{interpretación}" externa a $\mathfrak{U}$ de $X$.
        \begin{definition}[Universo]
            Un universo es un tipo $\mathfrak{U}$ junto con una familia
            $\mathfrak{U}$ sobre $\mathfrak{U}$ llamada 
            \textit{familia universal} tales que satisfacen los siguientes
            axiomas:

            \begin{itemize}
                \item $\mathfrak{U}$ es tal que existe 
                $$
                \check{\Pi} : \prod_{X : \mathfrak{U}} 
                    (\mathfrak{T}(X) \rightarrow \mathfrak{U}) 
                        \rightarrow \mathfrak{U}
                $$
                tal que satisface la siguiente igualdad de juicio:
                $$
                    \mathfrak{T}(\check{\Pi(X,Y)}) \equiv 
                        \prod_{x : \mathfrak{T}(X)} \mathfrak{T}(Y(x))
                $$
                para cualesquiera $X : \mathfrak{U}$ y 
                    $Y : \mathfrak{T}(X) \rightarrow \mathfrak{U}$.
                \item $\mathfrak{U}$ es tal que existe
                $$
                \check{\Sigma} : \prod_{X : \mathfrak{U}} 
                    (\mathfrak{T}(X) \rightarrow \mathfrak{U}) 
                        \rightarrow \mathfrak{U}
                $$
                tal que satisface la siguiente igualdad de juicio:
                $$
                    \mathfrak{T}(\Sigma(X,Y)) \equiv 
                        \sum\limits_{x : \mathfrak{T}(X)}\mathfrak{T}(Y(x))
                $$
                para cualesquiera $X : \mathfrak{U}$ y 
                $Y : \mathfrak{T}(X) \rightarrow \mathfrak{U}$.
                \item $\mathfrak{U}$ es tal que existe
                $$
                    \check{I} : \prod_{X : \mathfrak{U}} \mathfrak{T}(X)  
                        \rightarrow (\mathfrak{T}(X) \rightarrow \mathfrak{U})
                $$
                tal que satisface la siguiente igualdad de juicio:
                $$
                    \mathfrak{T}(\check{I}(X, x, y)) \equiv (x = y)
                $$
                para cualesquiera $X : \mathfrak{U}$ y $x,y : \mathfrak{T}(X)$.
                \item $\mathfrak{U}$ es tal que existe 
                $$
                    \check{+}:\mathfrak{U} \rightarrow \mathfrak{U} 
                        \rightarrow \mathfrak{U}
                $$
                tal que satisface la igualdad de juicio
                $$
                    \mathfrak{T}(X \check{+} Y) \equiv 
                        \mathfrak{T}(X) + \mathfrak{T}(Y)
                $$
                \item $\mathfrak{U}$ es tal que existen $\check{\mathbb{1}}, 
                \check{\mathbb{0}}, \check{\mathbb{N}}$ tales que satisfacen las 
                siguientes igualdades de juicio:
                \begin{align*}
                    \mathfrak{T}(\check{\mathbb{1}}) \equiv \mathbb{1}\\
                    \mathfrak{T}(\check{\mathbb{0}}) \equiv \mathbb{0}\\
                    \mathfrak{T}(\check{\mathbb{N}}) \equiv \mathbb{N}
                \end{align*}
            \end{itemize}
            
        \end{definition}

\subsection{Tipos primitivos}

\subsection{Tipos inductivos}

\subsection{Tipos de identidad}

\subsubsection{Aritmética modular}

\subsubsection{Equivalencia}
    \subsubsection{Equivalencias entre tipos}
\subsection{El teorema fundamental de los tipos de identidad}
\subsection{Proposiciones, conjuntos y niveles superiores de truncamiento}
\subsection{Extensionalidad de funciones}
\subsection{Truncamientos proposicionales}
\subsubsection{Lógica en teoría de tipos}
\subsection{Factorización de imágenes}
\subsection{Tipos finitos}
\subsection{El axioma de univalencia}
\subsection{Cocientes de conjuntos}

\bibliography{biblio}
\bibliographystyle{plain}

\end{document}

